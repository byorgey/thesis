%% -*- mode: LaTeX; compile-command: "mk" -*-

\begin{doublespace}

%%%%%%%%%%%%%%%%%%%%%%%%%%%%%%%%%%%%%%%%%%%%%%%%%%%%%%%%%%%%%%%%%%%%%%%%
%%%%%%%%%%%%%%%%%%%%%%%%%%%%%%%%%%%%%%%%%%%%%%%%%%%%%%%%%%%%%%%%%%%%%%%%
\begin{centering}
{\Large ABSTRACT} \\
\Title \\
Brent Abraham Yorgey \\
Stephanie Weirich \\
\end{centering}
%%%%%%%%%%%%%%%%%%%%%%%%%%%%%%%%%%%%%%%%%%%%%%%%%%%%%%%%%%%%%%%%%%%%%%%%
%%%%%%%%%%%%%%%%%%%%%%%%%%%%%%%%%%%%%%%%%%%%%%%%%%%%%%%%%%%%%%%%%%%%%%%%

\vspace*{1in}

The theory of \term{combinatorial species} was developed in the 1980s
as part of the mathematical subfield of enumerative combinatorics,
unifying and putting on a firmer theoretical basis a collection of
techniques centered around \term{generating functions}.  The theory of
\term{algebraic data types} was developed, around the same time, in
functional programming languages such as Hope and Miranda, and is
still used today in languages such as Haskell, the ML family, and
Scala.  Despite their disparate origins, the two theories have striking
similarities. In particular, both constitute algebraic frameworks in
which to construct structures of interest.  Though the similarity has
not gone unnoticed, a link between combinatorial species and algebraic
data types has never been systematically explored.  This dissertation
lays the theoretical groundwork for a precise---and, hopefully,
useful---bridge between the two theories.  One of the key
contributions is to port the theory of species from a classical,
untyped set theory to a constructive type theory. This porting process
is nontrivial, and involves fundamental issues related to equality and
finiteness; the recently developed \term{homotopy type theory} is put
to good use formalizing these issues in a satisfactory way.  In
conjunction with this port, species as general functor categories are
considered, systematically analyzing the categorical properties
necessary to define each standard species operation.  Another key
contribution is to clarify the role of species as \emph{labelled
  shapes}, not containing any data, and to use the theory of
\term{analytic functors} to model labelled data structures, which have
both labelled shapes and data associated to the labels.  Finally, some
novel species variants are considered, which may prove to be of use in
explicitly modelling the memory layout used to store labelled data
structures.

\end{doublespace}
